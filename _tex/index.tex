% Options for packages loaded elsewhere
% Options for packages loaded elsewhere
\PassOptionsToPackage{unicode}{hyperref}
\PassOptionsToPackage{hyphens}{url}
\PassOptionsToPackage{dvipsnames,svgnames,x11names}{xcolor}
%
\documentclass[
  ngerman,
  letterpaper,
  DIV=11]{scrartcl}
\usepackage{xcolor}
\usepackage{amsmath,amssymb}
\setcounter{secnumdepth}{5}
\usepackage{iftex}
\ifPDFTeX
  \usepackage[T1]{fontenc}
  \usepackage[utf8]{inputenc}
  \usepackage{textcomp} % provide euro and other symbols
\else % if luatex or xetex
  \usepackage{unicode-math} % this also loads fontspec
  \defaultfontfeatures{Scale=MatchLowercase}
  \defaultfontfeatures[\rmfamily]{Ligatures=TeX,Scale=1}
\fi
\usepackage{lmodern}
\ifPDFTeX\else
  % xetex/luatex font selection
\fi
% Use upquote if available, for straight quotes in verbatim environments
\IfFileExists{upquote.sty}{\usepackage{upquote}}{}
\IfFileExists{microtype.sty}{% use microtype if available
  \usepackage[]{microtype}
  \UseMicrotypeSet[protrusion]{basicmath} % disable protrusion for tt fonts
}{}
\makeatletter
\@ifundefined{KOMAClassName}{% if non-KOMA class
  \IfFileExists{parskip.sty}{%
    \usepackage{parskip}
  }{% else
    \setlength{\parindent}{0pt}
    \setlength{\parskip}{6pt plus 2pt minus 1pt}}
}{% if KOMA class
  \KOMAoptions{parskip=half}}
\makeatother
% Make \paragraph and \subparagraph free-standing
\makeatletter
\ifx\paragraph\undefined\else
  \let\oldparagraph\paragraph
  \renewcommand{\paragraph}{
    \@ifstar
      \xxxParagraphStar
      \xxxParagraphNoStar
  }
  \newcommand{\xxxParagraphStar}[1]{\oldparagraph*{#1}\mbox{}}
  \newcommand{\xxxParagraphNoStar}[1]{\oldparagraph{#1}\mbox{}}
\fi
\ifx\subparagraph\undefined\else
  \let\oldsubparagraph\subparagraph
  \renewcommand{\subparagraph}{
    \@ifstar
      \xxxSubParagraphStar
      \xxxSubParagraphNoStar
  }
  \newcommand{\xxxSubParagraphStar}[1]{\oldsubparagraph*{#1}\mbox{}}
  \newcommand{\xxxSubParagraphNoStar}[1]{\oldsubparagraph{#1}\mbox{}}
\fi
\makeatother


\usepackage{longtable,booktabs,array}
\usepackage{calc} % for calculating minipage widths
% Correct order of tables after \paragraph or \subparagraph
\usepackage{etoolbox}
\makeatletter
\patchcmd\longtable{\par}{\if@noskipsec\mbox{}\fi\par}{}{}
\makeatother
% Allow footnotes in longtable head/foot
\IfFileExists{footnotehyper.sty}{\usepackage{footnotehyper}}{\usepackage{footnote}}
\makesavenoteenv{longtable}
\usepackage{graphicx}
\makeatletter
\newsavebox\pandoc@box
\newcommand*\pandocbounded[1]{% scales image to fit in text height/width
  \sbox\pandoc@box{#1}%
  \Gscale@div\@tempa{\textheight}{\dimexpr\ht\pandoc@box+\dp\pandoc@box\relax}%
  \Gscale@div\@tempb{\linewidth}{\wd\pandoc@box}%
  \ifdim\@tempb\p@<\@tempa\p@\let\@tempa\@tempb\fi% select the smaller of both
  \ifdim\@tempa\p@<\p@\scalebox{\@tempa}{\usebox\pandoc@box}%
  \else\usebox{\pandoc@box}%
  \fi%
}
% Set default figure placement to htbp
\def\fps@figure{htbp}
\makeatother



\ifLuaTeX
\usepackage[bidi=basic]{babel}
\else
\usepackage[bidi=default]{babel}
\fi
% get rid of language-specific shorthands (see #6817):
\let\LanguageShortHands\languageshorthands
\def\languageshorthands#1{}
\ifLuaTeX
  \usepackage[german]{selnolig} % disable illegal ligatures
\fi


\setlength{\emergencystretch}{3em} % prevent overfull lines

\providecommand{\tightlist}{%
  \setlength{\itemsep}{0pt}\setlength{\parskip}{0pt}}



 
\usepackage[style=authoryear,]{biblatex}
\addbibresource{bib/bibliography.bib}


\KOMAoption{captions}{tableheading}
\makeatletter
\@ifpackageloaded{caption}{}{\usepackage{caption}}
\AtBeginDocument{%
\ifdefined\contentsname
  \renewcommand*\contentsname{Inhaltsverzeichnis}
\else
  \newcommand\contentsname{Inhaltsverzeichnis}
\fi
\ifdefined\listfigurename
  \renewcommand*\listfigurename{Abbildungsverzeichnis}
\else
  \newcommand\listfigurename{Abbildungsverzeichnis}
\fi
\ifdefined\listtablename
  \renewcommand*\listtablename{Tabellenverzeichnis}
\else
  \newcommand\listtablename{Tabellenverzeichnis}
\fi
\ifdefined\figurename
  \renewcommand*\figurename{Abbildung}
\else
  \newcommand\figurename{Abbildung}
\fi
\ifdefined\tablename
  \renewcommand*\tablename{Tabelle}
\else
  \newcommand\tablename{Tabelle}
\fi
}
\@ifpackageloaded{float}{}{\usepackage{float}}
\floatstyle{ruled}
\@ifundefined{c@chapter}{\newfloat{codelisting}{h}{lop}}{\newfloat{codelisting}{h}{lop}[chapter]}
\floatname{codelisting}{Listing}
\newcommand*\listoflistings{\listof{codelisting}{Listingverzeichnis}}
\makeatother
\makeatletter
\makeatother
\makeatletter
\@ifpackageloaded{caption}{}{\usepackage{caption}}
\@ifpackageloaded{subcaption}{}{\usepackage{subcaption}}
\makeatother
\usepackage{bookmark}
\IfFileExists{xurl.sty}{\usepackage{xurl}}{} % add URL line breaks if available
\urlstyle{same}
\hypersetup{
  pdftitle={„Über Menschen`` -- oder eine neue kognitive Komplexität zur Bewältigung von Ambiguität in Organisationen},
  pdfauthor={Sebastian Sukstorf},
  pdflang={de},
  pdfkeywords={Ambiguität, Führung, Organisation, Komplexität, kognitive
Flexibilität, Prototypenkultur, Leadership},
  colorlinks=true,
  linkcolor={blue},
  filecolor={Maroon},
  citecolor={Blue},
  urlcolor={Blue},
  pdfcreator={LaTeX via pandoc}}


\title{„Über Menschen`` -- oder eine neue kognitive Komplexität zur
Bewältigung von Ambiguität in Organisationen}
\author{Sebastian Sukstorf}
\date{2025-04-23}
\begin{document}
\maketitle
\begin{abstract}
Ambiguität ist kein Problem, sondern ein Gestaltungsraum. Der Text
zeigt, wie Führungskräfte lernen können, mit Mehrdeutigkeit konstruktiv
umzugehen -- durch Reflexion, Kontextbewusstsein und eine offene
Haltung. So wird Unsicherheit zur Quelle von Innovation und Vielfalt.
\end{abstract}

\renewcommand*\contentsname{Inhaltsverzeichnis}
{
\hypersetup{linkcolor=}
\setcounter{tocdepth}{3}
\tableofcontents
}

\section{Über Menschen}\label{uxfcber-menschen}

„Sie beharrt darauf, sich keine klare Meinung bilden zu müssen, wenn es
keine einfachen Lösungen gibt, und die gibt es momentan noch weniger als
sonst.`` So lässt Juli Zeh ihre Protagonistin Dora in dem Roman „Über
Menschen`` ihre Einschätzung zum Beginn der Corona-Pandemie beschreiben
\autocite[28.]{zeh2022}.

Mit diesem Zitat verdeutlicht Juli Zeh anschaulich, was Ambiguität im
Kern ausmacht. Nach Bauer werden mit dem Begriff \textbf{Ambiguität}
Zustände beschrieben, die als mehrdeutig oder weniger unterscheidbar
gelten. Sie erscheinen als vage und es sind Zustände, mit denen wir uns
als Menschen permanent auseinandersetzten müssen
\autocite[13]{bauer2018}.

Diese Auseinandersetzung erscheint für manche aber eine Zumutung und die
notwendige Ambiguitätstoleranz wird in vielen Lebensbereichen der
Gesellschaft immer geringer. Menschen streben nach Eindeutigkeit und die
notwendige Ambiguitätstoleranz ist scheinbar nur begrenzt ausgeprägt.
Der Wunsch nach Eindeutigkeit macht das Leben sicherlich einfacher, aber
er reduziert auch die Vielfalt \autocite{bauer2018}.

\section{Über Eindeutigkeit oder die latente Gefahr der
Blindheit}\label{uxfcber-eindeutigkeit-oder-die-latente-gefahr-der-blindheit}

Menschen unterscheiden sich in ihrer jeweiligen Ambiguitätstoleranz,
ähnlich wie im transaktionalen Umgang mit Stress \autocite{lazarus1984}.
In Organisationen haben Führungskräfte deswegen immer mehr die Aufgabe,
im Rahmen der allgemeinen Sorgfaltspflicht, die psychologischen
Einflüsse der MitarbeiterInnen und die Umwelt im jeweiligen System des
Unternehmens zu berücksichtigen und aktiv zu gestalten. Hierdurch tragen
sie aktiv dazu bei, die Ambiguitätstoleranz nicht nur zu fördern sondern
auch zu stärken. Ambiguitätsintoleranz und das Streben nach
Eindeutigkeit birgt eine große Gefahr: Wir verlieren die Vielfalt und
werden im wahrsten Sinne blind.

Diese ``Blindheit'' ist aber auch ein Zeichen von
ambiguitätsintoleranten Systeme. So tendieren Organisationen dazu,
Dinge, die nicht erkannt werden, als unwichtig zu betrachten oder ganz
fallen zu lassen \autocite[54]{bauer2018}.

Das zeigt sich beispielweise am Desinteresse vieler Menschen für
Politik. Das fehlende Interesse an politischen Themen und
Fragestellungen kann vielleicht ein Zeichen der Ambiguitätsverweigerung
sein und verweist auf eine mangelnde Bereitschaft zur inhaltlichen
Auseinandersetzung mit Politik \autocite[13 und 25]{bauer2018}.

Aber Juli Zeh beschreibt in ihrem Roman „Über Menschen`` nicht nur die
Situation zum Beginn der Corona-Pandemie als ein Bespiel für eine ambige
gesellschaftliche Situation, sie lässt Dora auch gleich eine Antwort für
eine mögliche Auflösung von Ambiguität geben:

\begin{quote}
„{[}\ldots{]} der Mensch kann viel weniger begreifen und kontrollieren,
als er glaubt. Auf dieses Dilemma kann weder Nichtstun noch Aktionismus
die richtige Antwort sein. Nach Doras Ansicht geht es um Augenmaß beim
Handeln und größtmögliche Ehrlichkeit in der Kommunikation.
Voraussetzung von Ehrlichkeit ist das Bekenntnis zum
Nicht-genau-Wissen.``
\end{quote}

Damit können hier schon die relevanten Handlungsfelder für einen
möglichen Abbau von Ambiguität ausgemacht werden. Es geht im
Wesentlichen um das eigene \textbf{Handeln}, unsere \textbf{Haltung} und
\textbf{Wahrnehmung}, im Sinne eines ‚Nicht-genau-Wissen`, sowie den
\textbf{Kontext}, in dem wir uns bewegen \autocite[vgl.][]{kozica2025}.

\section{Über das eherne
Ambiguitätsenergieerhaltungsgesetz}\label{uxfcber-das-eherne-ambiguituxe4tsenergieerhaltungsgesetz}

Aber eine Warnung gleich vorweg: Das Auflösen von Ambiguität erzeugt oft
neue Formen von Ambiguität. So schafft das Auflösen der ambigen Zustände
oft nie die gewünschte Eindeutigkeit. Ambiguität lässt sich vielleicht
gar nicht vollkommen auflösen. Bauer spricht in diesem Zusammenhang auch
von einem Ambiguitätsenergieerhaltungsgesetz:

\begin{quote}
„Je mehr Energie für die Beseitigung von Ambiguität aufgewendet wird,
desto mehr Ambiguität entsteht im Verhältnis zur jeweils beseitigten
Ambiguität.`` \autocite[76]{bauer2018}
\end{quote}

Uneindeutigkeit scheint also damit eine Zustand unserer Zeit zu sein. So
müssen wir uns nach Zygmut Bauman damit abfinden, dass Ambiguität das
Mittel der Moderne schlechthin ist, um unsere selbstverschuldeten und
destruktiven Potenziale aufzulösen \autocite{bauman2016}.

Der SAP-Vorstand Thomas Saueressig stellt in einem Interview fest, dass
Entscheidungen nie schwarz oder weiß sind und man mit dieser Ambiguität
umzugehen lernen muss. \autocite{witte2024} Aber manche Menschen lösen
diese Zustände nie auf, da sie Angst vor Ungewissheiten der Zukunft
haben und es ihnen am Vertrauen in die eigenen Möglichkeiten fehlt
\autocite[15]{morschitzky2009}.

Ambiguität, Undurchschaubarkeit und Unwissenheit über zukünftige
Entwicklungen und Paradoxien prägen den unternehmerischen Alltag. Vor
diesem Hintergrund lösen sich Gewissheiten auf und es reicht nicht mehr
aus, nur das „Bestehende zu optimieren``. Es gilt eine
„Sowohl-als-auch``-Haltung durch Reflexion der eigenen „Wahrnehmung,
Haltungen und Handlungen sowie den Kontext`` einzunehmen
\autocite{kozica2025}.

\section{Über inhärente Manager und die kognitive
Komplexität}\label{uxfcber-inhuxe4rente-manager-und-die-kognitive-komplexituxe4t}

Es werden also ambiguitätstolerante Führungskräfte benötigt, die nicht
bestrebt sind, Ambiguität zu vermeiden, sondern sie vielmehr suchen,
genießen und diese sehr bewusst zu gestalten \autocite[90]{bauer2018}.

Hruby schreibt diesen Menschen eine hohe kognitive Komplexität bei. Sie
zeichnen sich gerade dadurch aus, dass sie über eine „gewisse Struktur``
und über ausreichendes Wissen verfügen. Kognitiv komplexe Menschen
suchen nach umfangreichen und neuen Informationen. Sie verbringen viel
Zeit damit, die vorhandenen Informationen im gegebenen Kontext zu
interpretieren und betrachten diese auch von unterschiedlichen
Dimensionen. Eine hohe kognitive Komplexität kann dazu beitragen, mit
Ambiguität besser umzugehen \autocite[5]{hruby2014}.

Zu demselben Ergebnis kommt auch der von dem Beratungsunternehmen Profil
M und der Hochschule Fresenius publiziert Talent-Klima-Index {[}TKI{]}.
Der TKI 2024 nennt die Ambiguitätstolerant an dritter Stelle als eine
der wichtigsten Kompetenzen von Führungskräften, nach Optimismus,
Zukunftsorientierung und einem positiven Umgang mit Veränderungen
\autocite{stulle2024}.

\section{Über Kohärenz- oder
Komplexitätsmaschinen}\label{uxfcber-kohuxe4renz--oder-komplexituxe4tsmaschinen}

Aber es darf berechtigterweise gefragt werden, ob Führungskräfte mit
diesen Fähigkeiten auch im System ihrer Organisation über die nötigen
Zeitressourcen verfügen, um mit ambigen Zuständen umgehen zu können.

Vielleicht können die handelnden Personen im Wirtschaftssystem von
anderen System lernen. So wagt Gutzmer einen Blick auf den Architekten
Rem Koolhaas vom \href{https://www.oma.com}{OMA -- Office for
Metropolitan Architects} und stellt die provokante Frage, ob Manager
eventuell etwas von Architekten lernen können.

Architekten, wie Manager waren in der Vergangenheit Kohärenzmaschinen,
mit einem allwissenden und vorausschauenden Anspruch zur sicheren
Gestaltung einer eindeutigen Welt. Dieser Zustand löst sich aufgrund der
zunehmenden Geschwindigkeit bei den vielfältigen und gleichzeitigen
Veränderungen aber immer mehr auf \autocite[11]{roth2019}. Stattdessen
ist es dem inhärenten Manager auferlegt, mit den Widersprüchlichkeiten
seiner Zeit im Zusammenhang seiner Rolle(n) immer wieder neue und offene
Lösungen zu finden \autocite[102]{gutzmer2020}.

In diesem Zusammenhang können kognitiv flexible und inhärenten Manager
sich an den Diagramm-Modellen der Architekten von OMA orientieren. Diese
Diagramme, sind ein visuelles Rechenergebnis, welches alle Ergebnisse,
Vermutungen und jegliches Wissen im System eines Gebäude integrierte.

Damit werden diese Diagramme ein Kommunikationsmittel zur Erläuterung
der ambigen Zustände und integrieren alle vorherrschenden
Widersprüchlichkeiten in einem Gebäude oder einem System. Es gilt also
auch im unternehmerischen Kontext die richtigen Diagramm-Modell zu
finden und zu verwenden, um in immer komplexeren Kommunikationen zu
agieren.

Die oftmals in Unternehmen vorhandene Visions-, Leitbilder oder
Führungsleitlinien sind einzelne Kommunikationsmedien, die entweder von
der Geschäftsführung und Führungskräften, der Personalabteilung oder in
der Marketingabteilung oftmals losgelöst entwickelt wurden. Diese gut
gemeinten Guidelines ähneln - um im Bilder der Architektur zu bleiben -
eher den unterschiedlichen Materialien auf einer Großbaustelle: Lose
hingeworfene Sandhaufen, kunterbunte Steinwüsten oder scharfkantig
gestapelte Metallträger. Es fehlt an einem integrierenden System,
welches `Nicht-genau-Wissen' integriert.

Diese Widersprüchlichkeit fehlt oft bei dem Verfolgen einer klaren
Vision und dem dauerhaften Beibehalten von Kommunikationslinien über
längere Zeiträume hinweg. So schränken sie die vorhandenen
Möglichkeitsräume ein und schaffen eine trügerische Eindeutigkeit auf
Kosten einer notwendigen Innovation und Offenheit für Ambiguität in
Absatzmärkten und damit für die Bedürfnisse der Kunden.

So entwickeln sich Manager immer mehr zu ambiguitätstoleranten
Persönlichkeiten, die nicht bestrebt sind, Ambiguität zu vermeiden,
sondern diese suchen, genießen und aktiv gestalten. Manager werden so zu
„fehlerproduzierenden Ambiguitäts- und Komplexitätsmaschinen``
\autocite{gutzmer2020}.

\section{Über was gilt es
nachzudenken}\label{uxfcber-was-gilt-es-nachzudenken}

Im folgenden sind vier Bereich mit relevanten Aspekten zur Reflexion
über die Ambiguitätstoleranz zusammengefasst. Sowohl Führungskräfte als
auch Teams können sich diese Fragen beantworten.

\begin{itemize}
\tightlist
\item
  \textbf{Handeln}

  \begin{itemize}
  \tightlist
  \item
    Strukturelle Faktoren für unser Handeln hinterfragen
  \item
    Machtfragen erkennen und verstehen {[}ggf. verändern und neu
    gestalten{]}
  \item
    Individuelle Handlungsstrategien kennen, regelmäßig überprüfen und
    ggf. anpassen
  \end{itemize}
\item
  \textbf{Haltung}

  \begin{itemize}
  \tightlist
  \item
    Selbstüberprüfung und -reflexion der eigenen inhärenten Rolle und
    Person
  \item
    Gestaltung einer Prototypenkultur und Reduktion von
    statuskonstruierenden Wissensansprüchen
  \end{itemize}
\item
  \textbf{Wahrnehmung}

  \begin{itemize}
  \tightlist
  \item
    Überprüfung der Denkprozesse auf individueller Ebene
  \item
    Gelassenheit in Führungs- und Visionsprozessen
  \end{itemize}
\item
  \textbf{Kontext}

  \begin{itemize}
  \tightlist
  \item
    Hinterfragen der Beziehungen zur sozialen Umwelt
  \item
    Verständnis schaffen für die Art und Weise, wie Erfahrungen erklärt
    werden
  \item
    Widersprüchlichkeit als Element von visionärem Denken und Innovation
    verstehen
  \end{itemize}
\end{itemize}

\section{Über ein Fazit}\label{uxfcber-ein-fazit}

Ambiguität ist kein Problem, das gelöst werden kann -- sie ist vielmehr
ein Zustand, den es zu gestalten gilt. Führungskräfte und Organisationen
müssen lernen, Mehrdeutigkeit nicht nur auszuhalten, sondern sie aktiv
als Chance für Innovation und Vielfalt zu nutzen. Nur so können sie den
Herausforderungen einer immer komplexeren Welt gerecht werden.

\section{Literaturverzeichnis}\label{literaturverzeichnis}

\printbibliography[heading=none]





\end{document}
